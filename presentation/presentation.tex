\documentclass{beamer}

\usetheme{Pk}

\newcommand\rb[1]{\textcolor{ThemeRed}{\textbf{#1}}}
\usepackage{listings}
\lstset{
  basicstyle=\scriptsize,
  columns=fullflexible,
  showstringspaces=false,
  commentstyle=\color{ThemeGrey}\upshape
}

\lstdefinelanguage{XML}
{
  morestring=[b]",
  moredelim=[s][\color{ThemeGreen}]{<}{\ },
  moredelim=[s][\color{ThemeGreen}]{</}{>},
  moredelim=[l][\color{ThemeGreen}]{/>},
  moredelim=[l][\color{ThemeGreen}]{>},
  stringstyle=\color{ThemeBlue},
  identifierstyle=\color{ThemeGreen},
  keywordstyle=\color{ThemeRed},
}

\hypersetup{
  colorlinks = true,
  linkcolor = ThemeGrey,
  urlcolor = ThemeLightBlue
}   


\title{Open Data Management \& Cloud\\exam project}
\subtitle{Audio music file archiving}
\author{Patrick Indri}
\date{\today}


\begin{document}
	\setcounter{showSlideNumbers}{0}

	\frame{\titlepage}

	\setcounter{framenumber}{0}
	\setcounter{showSlideNumbers}{1}



\section{Introduction}

  \begin{frame}
    \frametitle{Structure of the project}
  \end{frame}



  \begin{frame}
    \frametitle{Introduction}

    \begin{block}{Aim of the project}
      Investigation of audio file archiving for music.
    \end{block}

    \vspace{1em}

    In particular:
    \begin{itemize}
      \item UML metadata model;
      \item XSD implementation and XML sample document;
      \item discussion of data discovery/access and interoperability;
      \item discussion of long term archiving and data preservation.
    \end{itemize}

    \vspace{1em}

    \rb{Data resource:} not an actual dataset but music files in general.

  \end{frame}



  \begin{frame}
    \frametitle{Metadata standards for audio files}

    There is no widely used and standardised metadata model for music audio files.

    \vspace{1em}

    \begin{itemize}
      \itemsep1em
      \item Dublin Core: simple (15 terms), focus on descriptive metadata;
      \item EbuCore: detailed DC extension, fine grain technical and administrative metadata for broadcasting;
      \item METS: handles the structural/hierarchical metadata of a digital library. Open flexibility (no vocabulary).
    \end{itemize}
  \end{frame}



\section{Model design and implementation}


  \begin{frame}
    \frametitle{Model design}

    What should the data model represent?

    \vspace{1em}

    \begin{itemize}
      \itemsep0.75em
      \item Songs and their different versions;
      \item Groups of songs (releases);
      \item Artists;
      \item Basic technical metadata;
      \item Relations between songs, releases and artists.
    \end{itemize}

  \end{frame}



  \begin{frame}
    \frametitle{UML}
    \begin{center}
      \includegraphics[width=0.95\textwidth]{img/UML.pdf}
    \end{center}
  \end{frame}



  \begin{frame}
    \frametitle{XSD}

    Choice of implementation:

    \vspace{0.5em}

    \begin{itemize}
      \item RDB: easy to enforce constraints (primary/foreign keys), widely used, easy to model relationships, rigid structure;
      \item \rb{XSD}: flexible, easily handle partial data, harder relationship handling.
    \end{itemize}

    \vspace{1.5em}

    The proposed XSD implementation should:

    \vspace{0.5em}

    \begin{itemize}
      \item Refine Dublin Core;
      \item Balance integrity constraints and partial data;
      \item Model relationships with detail. 
    \end{itemize}

    \vspace{1.5em}

    The resulting XSD can be retrieved \href{https://github.com/pindri/ODMC_exam/blob/master/odmc.xsd}{here}.
  \end{frame}




  \begin{frame}[fragile]

    Example of an XML document, valid against the proposed XSD.

    \vspace{1em}

    \frametitle{XML example}
\lstset{basicstyle=\scriptsize}
\begin{lstlisting}[language=XML]
<work>
  <ISWC id="ISWC_T-000.000.000-A"></ISWC>
  <title lang="en">
    <dc:title>Test Work</dc:title>
  </title>
  <hasArtist label="Will Wilson" description="Singer">
  </hasArtist>
  <hasPerformance label="Test Rec." description="Studio Ver.">
    <relationIdentifier>
      <ISRC idref="ISRC_AAAAA0000000"></ISRC>
    </relationIdentifier>
  </hasPerformance>
</work>
\end{lstlisting}
  \end{frame}



  \begin{frame}
    \frametitle{Difficulties and possible expansions}
  \end{frame}



\section{Interfaces and Services}

  \begin{frame}
    \frametitle{Data discovery: search/filter service}
  \end{frame}



  \begin{frame}
    \frametitle{Data annotation}
  \end{frame}



  \begin{frame}
    \frametitle{Storage and Cloud Solutions}
  \end{frame}



\section{Preservation and Interoperability}

  \begin{frame}
    \frametitle{Data preservation}
  \end{frame}


  
  \begin{frame}
    \frametitle{Interoperability}
  \end{frame}

  

  \begin{frame}
    \frametitle{Audio file formats}
  \end{frame}

  

\section{Final considerations}

  \begin{frame}
    \frametitle{Final considerations}
  \end{frame}



\backupbegin

\section{References}

  \begin{frame}
    \frametitle{References}
  \end{frame}

\backupend

\end{document}
